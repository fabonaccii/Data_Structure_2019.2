\documentclass[a4paper,12pt]{article}

\usepackage{tikz}

% Reference

\usepackage[utf8]{inputenc}

\usepackage[lmargin=2cm,tmargin=2cm,rmargin=2cm,bmargin=2cm]{geometry}

\usepackage[onehalfspacing]{setspace}

\usepackage[T1]{fontenc}

\usepackage[brazil]{babel}

\usepackage{graphicx, xcolor, comment, enumerate, multirow, multicol, indentfirst,}

\usepackage{amsmath,amsthm,amsfonts,amssymb,dsfont,mathtools,blindtext}


\begin{document}

\begin{center}

\begin{figure}

    \centering
    
    \includegraphics[scale=1.5]{UFC.png}

\end{figure}

\textbf{\Huge{Relatório - Árvore de Ordenação}}
\vspace{0.5cm}

\textbf{\Huge{Estrutura de Dados}}
\vspace{1cm}
    
    \Large {
    Professor: Atílio Gomes Luiz
    
    Aluno: Fábio Luz Duarte Filho
    
    Matrícula: 474027
    
    Ciência da Computação
    
    Universidade Federal do Ceará - Quixadá/CE - Brasil
    }
    
\end{center}

\vspace{1cm}

\section{Listagem dos programas em C++}
    \subsection{Tree.h}
        Contém o struct TreeNode e os cabeçalhos das funções que envolvem a árvore.
        
    \subsection{Tree.cpp}
        Contém as implementações das funções iniciadas em Tree.h.
        
    \subsection{Main.cpp}
        Contém a função main, a qual pertence o código executável que utiliza as funções da Tree.h para realizar a ordenação em árvore.
        
\section{Listagem dos testes executados}
    \subsection{Entradas}
        . \\
        5 \\
        8 20 41 7 2 \\
        10 \\
        23 3 45 6 1 9 37 99 0 30 \\
        3 \\
        345 2 1 \\
        5 \\
        98 34 2 1 76 \\
        0 
        
    \subsection{Saídas}
        . \\
        2 7 8 20 41 \\
        0 1 3 6 9 23 30 37 45 99 \\
        1 2 345 \\
        1 2 34 76 98

\section{Descricões}

    \subsection{Estruturas}
        \begin{list}{} {
        \setlength{\leftmargin}{0cm}
        \setlength{\rightmargin}{0cm}
        \setlength{\labelwidth}{0pt}
        \setlength{\labelsep}{\leftmargin}}
            \item Nós de árvore (TreeNode's) implementados em struct.\
            \item 1 vetor de inteiro para conter as entradas múltiplas.\
            \item 2 vetores de ponteiro de TreeNode (TreeNode*) para manipular os nós da árvore.\
        \end{list}
        
    \subsection{Decisões e especificações}
        O programa captura, para cada caso, o tamanho do vetor, checa se esse é uma potência de dois (já que o número de folhas para que a árvore seja cheia precisa ser uma potência de dois). O vetor é criado com tamanho igual à potência de dois, mais próxima, maior ou igual ao tamanho captado. O vetor é preenchido com a segunda linha captada. Caso o tamanho do vetor captado não seja uma potência de dois, o vetor é preenchido com um epsilon(E) que é maior que todos os números do vetor.
        \\
        
        Vetores de TreeNode* são criados para conter a árvore e manipular seus nós. Um deles sofrerá alterações de acordo com a comparação de nós. O outro, que serve de backup, sofre apenas uma alteração por ciclo, a qual substitui a menor folha por epsilon. A cada ciclo, o primeiro vetor recebe os nós do vetor backup. Ao fim de cada ciclo, o menor valor da árvore é imprimido no arquivo de saída.
        \\
        
        Ao fim de cada caso passado na entrada, a árvore criada é liberada e busca-se o próximo caso, até que a entrada do tamanho do vetor seja igual a zero, caso no qual o programa encerra.
    
\section{Complexidades}

    \subsection{Funções}
    
        \subsubsection{TreeNode* createNode}
            Implementada em Tree.cpp. Possui complexidade O(1), visto que apenas cria um nó alocado e "seta" os campos da struct. 
        
        \subsubsection{TreeNode* freeTree}
            Implementada em Tree.cpp. Possui complexidade O(n), visto que percorre a árvore recursivamente deletando os nós de baixo para cima.
            
        \subsubsection{int treeHeight}
            Implementada em Tree.cpp. Possui complexidade O(1), visto que apenas calcula a altura da árvore cheia de acordo com a quantidade de folhas dela.
            
        \subsubsection{int getKey}
            Implementada em Tree.cpp. Possui complexidade O(1), visto que apenas retorna a chave de certo nó repassado com parâmetro.
            
        \subsubsection{void setKey}
            Implementada em Tree.cpp. Possui complexidade O(1), visto que apenas "seta" uma chave em um nó, ambos passados como parâmetro.
            
        \subsubsection{TreeNode* nodesComparisonToCreate}
            Implementada em Tree.cpp. Possui complexidade O(1), visto que apenas compara as chaves de dois nós passados como parâmetros e cria um nó que será pai desses dois nós contendo a menor chave entre os nós.
            
        \subsubsection{TreeNode* whichNodeIsSmaller}
            Implementada em Tree.cpp. Possui complexidade O(n), visto que percorre um vetor de TreeNode's passado como parâmetro a fim de encontrar e retornar o nó que possui menor chave.
            
        \subsubsection{int whichIsBigger}
            Implementada em Main.cpp. Possui complexidade O(n), visto que percorre um vetor de inteiros passado como parâmetro a fim de encontrar e retornar o índice que possui o inteiro de maior valor. 
    
    \subsection{Programa}
        O programa possui complexidade O($n^2$), visto que possui vários for's e while's aninhandos em um único while "global" para cada caso de entrada. Ainda que sejam alguns casos, o programa permanece O($n^2$). 

\end{document}
